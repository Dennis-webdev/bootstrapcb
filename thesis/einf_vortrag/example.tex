% This example is meant to be compiled with lualatex or xelatex
% The theme itself also supports pdflatex
\PassOptionsToPackage{unicode}{hyperref}
\documentclass[aspectratio=1610, 9pt]{beamer}

% Load packages you need here
\usepackage{polyglossia}
\setmainlanguage{german}

\usepackage{csquotes}
    

\usepackage{amsmath}
\usepackage{amssymb}
\usepackage{mathtools}

\usepackage{hyperref}
\usepackage{bookmark}

% mine
\usepackage{algpseudocode}

% load the theme after all packages

\usetheme[
  showtotalframes, % show total number of frames in the footline
]{tudo}

% Put settings here, like
\unimathsetup{
  math-style=ISO,
  bold-style=ISO,
  nabla=upright,
  partial=upright,
  mathrm=sym,
}

\title{Bootstrapping Ansätze zur Bestimmung von Konfidenzbändern für Verteilungsfunktionen}
\author[D.~Richter]{Dennis Richter}
\institute[LS 4]{Lehrstuhl IV \\ Informatik}
\titlegraphic{\includegraphics[width=0.55\textwidth]{images/tudo-title-2.jpg}}


\begin{document}

\maketitle

\begin{frame}
  \frametitle{Motivation und Problemstellung}
  \begin{center}
    \includegraphics[width=0.7\textwidth]{images/4.png}
  \end{center}
\end{frame}

\begin{frame}
  \frametitle{Motivation und Problemstellung}
  \begin{itemize}
    \item Konfidenzintervalle sind visuelle Hilfestellung zum interpretieren der Daten
    \item die Intervalle gelten jeweils immer nur für einen Messwert
    \item manchmal möchte man solche Schätzungen für die ganze Regressionsfunktion haben, dann benötigt man Konfidenzbänder
    \item Es gibt viele Verschiedene Methoden zu Bestimmung von Konfidenzbändern
    \item neben analytischen Verfahren, kann auch Bootstrap eingestetzt werden -> einfacher, aber ähnlich gut
    \item zwar theoretische ansätze in der Literatur aber wenig tatsächliche Implementierungen und damit Emipirischer Vergleich zu herkömlichen Methoden
    \item (Aufgabe ist also vorstellung der Methoden, umsetztung in Omnet und empirischer Vergleich zu anderen CBs)
  \end{itemize}
\end{frame}

%\begin{frame}
%  \frametitle{PIBA - vier Stichpunkte (anstatt Gliederungsfolie)}
%\end{frame}

\begin{frame}
  \frametitle{Grundlagen}
  \begin{itemize}
    \item Regressionsfunktion $\eta(x, \theta)$: \\ 
    gesucht ist $\theta_0$ mit $y_j = \eta(x_j, \theta_0) + \epsilon_j$, $j = 1,2...,n$ und $\epsilon \sim N(0, \sigma^2)$ \\
    \item Schätzer $\hat{y}(x) = \eta(x, \hat\theta)$ für den "wahren Wert" $E(y|x) = \eta(x, \theta_0)$ \\
    \item Konfidenzintervall für den Schätzer: \\
    $P\left( \theta_L \le \theta_0 \le \theta_U \right) \geq 1-\alpha$
    \\
    z.B. $\theta_L, \theta_U = \hat\theta \mp z_{\alpha/2}\sqrt{\mathbf{V}(\hat\theta)}$
    \\
    \item Konfidenzintervall für einen Punkt der Regressionsfunktion: \\
    $\forall x: P\left(y_L(x) \le \eta(x, \theta_0) \le y_U(x)\right) \geq 1-\alpha$
    \\
    z.B. $y_L(x), y_U(x) = \eta(x, \hat\theta) \mp 
    z_{\alpha/2}
    \sqrt{
      \left(
        \frac{\partial\eta(x, \theta)}{\partial\theta}
      \right)_{\hat\theta}^T
      \mathbf{V}(\hat\theta)
      \left(
        \frac{\partial\eta(x, \theta)}{\partial\theta}
      \right)_{\hat\theta}
    }$
    \item Konfidenzband für die Regressionsfunktion: \\
    $P\left(\forall x: y_L(x) \le \eta(x, \theta_0) \le y_U(x) \right ) \geq 1-\alpha$
    \\
    z.B. $y_L(x), y_U(x) = \eta(x, \hat\theta) \mp 
    \sqrt{
      \chi_p^2(a)
      \left(
        \frac{\partial\eta(x, \theta)}{\partial\theta}
      \right)_{\hat\theta}^T
      \mathbf{V}(\hat\theta)
      \left(
        \frac{\partial\eta(x, \theta)}{\partial\theta}
      \right)_{\hat\theta}
    }$
  \end{itemize}
\end{frame}

\begin{frame}
  \frametitle{Grundlagen}
  \hspace{20px} \vline \hspace{5px}   
  \begin{minipage}[t]{0.3\linewidth}
    Basic-Sampling-Methode: \\
    \begin{algorithmic}
			\For {$j=0$ to $B$}
				\For {$i=0$ to $n$} 
				  \State Draw sample $y_{ij}$ from $F(.)$
				\EndFor
				\State Calculate statistic $s_j = s(y_j)$
			\EndFor
			\State Form the EDF $G_n(.|s)$
	  \end{algorithmic}
  \end{minipage}
  \hspace{10px} \vline \hspace{5px}
  \begin{minipage}[t]{0.48\linewidth}    
    Bootstrap-Methode: \\
    \begin{algorithmic}
      \Require Random sample $y = (y_1, y_2, ...y_n)$ from $F(.)$
      \State Form the EDF $F_n(.|y)$
			\For {$j=0$ to $B$}
				\For {$i=0$ to $n$} 
				  \State Draw sample $y^*_{ij}$ from $F_n(.|y)$
				\EndFor
				\State Calculate statistic $s^*_j = s(y^*_j)$
			\EndFor
			\State Form the EDF $G_n(.|s*)$
	  \end{algorithmic}
  \end{minipage}
\end{frame}

\begin{frame}
  \frametitle{Verwandte Arbeiten} 
  \begin{itemize}
    \item Cheng, Russell. (2005). Bootstrapping simultaneous confidence bands. 8 pp.-. 10.1109/WSC.2005.1574257.
    \item Cheng, Russell. (2015). Bootstrap confidence bands and goodness-of-fit tests in simulation input/output modelling. 16-30. 10.1109/WSC.2015.7408150.
    \begin{center}
      \includegraphics[width=0.5\textwidth]{images/3.png}
    \end{center}
  \end{itemize}
  Weitere:
  \begin{itemize}
    \item Govind, Nirmal \& Roeder, Theresa. (2006). Estimating Expected Completion Times with Probabilistic Job Routing. 1804-1810. 10.1109/WSC.2006.322958. 
    \item Wang, Xing \& Wang, Xin \& Sun, Zhaonan. (2009). Comparison on Confidence Bands of Decision Boundary between SVM and Logistic Regression. 272-277. 10.1109/NCM.2009.281.
  \end{itemize}
\end{frame}

\begin{frame}
  \frametitle{Lösungsansatze}
  In den Papern werden 2 Lösungsansätze vorgestellt, die Bootstrap zur Vereinfachung der erwähnten analytischen Verfahren einsetzten
  \begin{itemize}
    \item Parametric Bootstrap:
    \begin{itemize}
      \item setzt voraus, dass $\hat\theta$ als normalverteilt angenommen werden kann, also $\hat\theta \sim N(\theta_0, \mathbf{V}(\theta_0))$
      \item verzichtet allerdings auf lineare Approximation von $\eta(x,\theta)$ durch die Delta-Methode
    \end{itemize}
    \item Non-Parametric Bootstrap:
    \begin{itemize}
      \item keine Verteilungsannahme über $\hat\theta$
      \item und auch keine lineare Approximation von $\eta(x,\theta)$
      \item rechenintensiv, da geschachteltes Bootstrap 
    \end{itemize}
  \end{itemize}
  Es gibt allerding auch andere analytische Verfahren zur Bestimmung von Konfindenzbändern, bei denen Bootstrap zum Einsatz kommen kann.
\end{frame}

\begin{frame}
  \frametitle{Anwendnungs Beispiel}
  \begin{minipage}[t]{0.48\linewidth}
	  \begin{figure}
	    \includegraphics[width=0.8\textwidth]{images/5.png}
	    \caption{Morocco Pulmonary TB notifications per 100,000}
	  \end{figure}
  \end{minipage} 
  \begin{minipage}[t]{0.48\linewidth}
	  \begin{figure}
	    \includegraphics[width=0.4\textwidth]{images/6.png}
	    \caption{MLE’s for the Morocco TB Model}
	  \end{figure}
	\end{minipage}
	\begin{figure}
	  Als Modell wurde gewählt: 
		\begin{align*}
		    y_j = (\theta_2 + \theta_4 x_j + \theta_6 x^2_j)
			  \frac{
			    exp(\theta_5(x_j - \theta_3))}{
			    1 + exp(\theta_5(x_j - \theta_3))
			  } + \epsilon_j
		\end{align*}
		wobei $\epsilon_j \sim N(0, \theta_1^2)$
	\end{figure}
\end{frame}

\begin{frame}
  \frametitle{Anwendnungs Beispiel}
  \begin{center}
    \includegraphics[width=0.6\textwidth]{images/1.png}
  \end{center}
\end{frame}

\begin{frame}
  \frametitle{Umsetzung}
  \begin{itemize}
    \item genauere Recherche zu Konfidenzbändern und Bootstrapping-Ansätze zur Bestimmung von Konfidenzbändern (mehr dimensionale Parameter)
    \item Recherche zu Parameterstudien, Auswertung und Darstellung in Kontext von OMNeT++
    \item Erstellung einfacher Beispiele als Grundlage (z.B. für die Evaluation)
    \item Implementierung der Verfahren in C++ und Integration in die OMNeT++ IDE
    \item Anwendung der Verfahre anhand der erstellten Beispiele
  \end{itemize}
\end{frame}

\begin{frame}
  \frametitle{Geplante Evaluation}
  \begin{itemize}
    \item 
  \end{itemize}
\end{frame}

\begin{frame}
  \frametitle{Zeitplan}
  \begin{center}
    \includegraphics[width=0.8\textwidth]{images/2.png}
  \end{center}
\end{frame}

%	\begin{frame}
%	  \frametitle{PIBA - vier Stichpunkte (anstatt Gliederungsfolie)}
%	  \begin{itemize}
%	    \item Was ist das Problem? -> Implementierung von Bootstrap-Ansätzen zur Bestimmung von Konfidenzbändern
%	    \item Idee? -> Implementierung einiger Ansätze im Kontext von OMNeT++, Integration in die IDE nach Möglichkeit und Bewertung anhand von Simulationen
%	    \item Vorteil? -> Boostrap ist ein simples und allgemeines Verfahren und einsetzbar in ungewissen situationen -> die arbeit gibt einen erfahrungswert
%	    \item Aktion? -> zuerst die genannten und weitere Ansätze genauer recherchieren und vorstellen, dann in C++ im Kontext von OMNeT++ implementieren und nach Möglichkeit in die IDE integrieren, dann testen der Methoden anhand von einfachen Beispielen in Form einer OMNeT++ Simulation (z.B. ...)
%	  \end{itemize}
%	\end{frame}
\end{document}