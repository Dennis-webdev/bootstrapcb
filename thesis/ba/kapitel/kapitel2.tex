% kapitel2.tex
\chapter{Grundlagen}
\label{chapter:kap2}

Zuerst besprechen wir ein paar Grundlagen, die notwendig für die folgenden Kapitel sind

\section{Simulationsstudien}
Eine wesentliche Aufgabe der Statistik ist das Analysieren von realen System, um exakte Informationen zu relevanten Fragen zu erhalten.

Dabei kann es sich zum Beispiel um die Optimierung eines Herstellungsprozesses handeln.

Der Prozess in diesem Fall wird System genannt und Ziel der Analyse ist das Verhalten des Systems unter Veränderung der Eingabeparameter zu bestimmen.

Es gibt verschiedene Wege das Verhalten eines Systems zu studieren, Law... gibt einen guten Überblick zu diesem Thema.

Verschiedene Möglichkeiten ein System zu studieren werden von Law in folgender Grafik dargestellt.

Grundsätzlich kann man unterscheiden zwischen Experimenten an dem echten System und an einem Model des Systems. 

Experimente an dem echten System würden bedeuten, dass man die physikalischen Gegebenheiten des System nach belieben verändern kann und die Daten für die Analyse anhand des realen Systems erheben kann

Dieser Ansatz ist immer erstrebenswert, da Unsicherheiten in der Wahl des Modells komplett eliminiert werden.

Allerdings ist dies nur in den seltensten Fällen möglich, da man in der Regel nur begrenzte Material- oder Geldressourcen zur Verfügung hat oder das System durch die Änderung zu sehr gestört wird.

Die am meisten Eingesetzte Variante ist die Simulation, bei der Versucht wird das reale Model so genau wie möglich mathematisch abzubilden und durch Computer gestützte Simulation beliebig viele Daten zu erheben.

Die Versuche der Simulation sind iid was für die späteren Methoden ein wesentlicher Faktor ist

Ein Modell welches das reale Modell repräsentiert wird Simulationsmodell genannt und kann beliebig komplex sein. 

In unserem Fall betrachten wir ...

Ein einfaches Simulationsmodell welches eine ...Queue repräsentiert ist das M/M/1-Modell, diese sind sehr Verbreitet in der Statistik und gut bereits recherchiert.

Law gibt eine gute Einführung und unser Szenario orientiert sich stark daran.

Nachdem die Daten durch die Simulation erhoben wurden, wollen wir das Verhalten des Systems unter Veränderung der Parameter (etwa. der arivalrate oder servicetime) analysieren und durch eine Formel angeben.

Solch eine Formel die wiederum das Simulationsmodell vereinfacht repräsentiert ist auch ein mathematisches Modell und wird Metamodell genannt (bzw. Regressions Metamodell falls...)

Barton gibt eine gute Einführung für die Bestimmung solcher Metamodelle

Wir nehmen an, dass das Simulationsmodell das reale Modell korrekt repräsentiert, das Metamodell ist dann ... in Abhängigkeit von einem Parametervektor

Die Berechnung der Regressionsfunktion und Analyse der Methoden ist für uns primäres ziel der Simulationsstudie

Angenommen das Modell repräsentiert die Simulation und es ex. ein wahrer wert theta0, dann ist das erste Problem diesen zu bestimmen bzw zu schätzen

Die mit Abstand effektivste Methode theta in parametrischen Studien zu schätzen ist die maximum likelyhood methode, welche im nächsten abschnitt kurz vorgestellt werden soll

\subsection{M/M/1-Modell}

%%%%%%%%%%%%%%%%%%%%%%%%%%%%%%%%%%%%%%%%%%%%%%%%%%%%%
% Quellen 

Barton,  R.R.,  1998.  Simulation  metamodels:

Law,  A.M.  and  Kelton,  W.D.  1991.  Simulation  modeling  and analysis:

%%%%%%%%%%%%%%%%%%%%%%%%%%%%%%%%%%%%%%%%%%%%%%%%%%%%%

\begin{equation}
y_j = \eta(x_j, \theta_0) + \epsilon_j, 
\quad j=1,2,...,n \text{ und } \epsilon \sim N(0, \sigma^2)
\end{equation}

- Quellen: Barton 1998, Krazanowiski 1998\\
- geben sei eine situation, wo ein regression metamodel als repräsentation fpr die ausgabe einer simulations studie verwendet wird\\
- n unabhängige versuche und die beobachteten werte des simulations models seien gegeben durch y\\
- y ist zufallsvariable abhängig von einem design punkt x\\
- eine gewöhnliche darstellung ist durch ... (means of a statistical metamodel)\\
- ein paar beispiele die zur orientierung dienten geben\\
- eta bezeichnet eine deterministische funktion, namentlich regressionsfunktion\\
- genauer gesagt ist eta ein parmetrisches statistisches metamodel\\
- es wird angenommen, dass die simulation durch dieses modell repräsentiert werden kann und der erwartungswert dieses metamodels spiegelt dann den wahren erwartungswert wieder, vorausgesetzt die annahme trifft zu\\
- epsilon bezeichnet den für alle design punkte unabhängigen zufallsfehler mit mean 0\\
- oft wird angenommen dass die varianz für alle design punkte gleich ist, aber nicht zwingend\\
- jedoch mean(epsilon)=0 bedeutet eta gibt den erwartungs wert der statistik an\\


\section{Parameter-Schätzer}
Hat man ein mathematisches Modell (bzw. regression metamodell) bestimmt, welches die Verteilung einer Datenmenge beschreiben soll, ist die Frage nun, was ist der wahre Wert des Parameters unter der Annahme, dass das Modell korrekt ist?

Der Parameter ist eine Zufallsvariable, die unter sehr allgemeinen Voraussetzungen multivariat normalverteilt um den wahren Wert ist.



Bei der linearen Regression wir oft die Kleinste-Quadrate-Methode verwendet. 

%%%%%%%%%%%%%%%%%%%%%%%%%%%%%%%%%%%%%%%%%%%%%%%%%%%%%
% Quellen
Chernick, M.R. 1999.Bootstrap Methods, A Practitioners Guide 3.1.1.

Efron, B. (1982) The Jackknife, the Bootstrap and Other Resampling Plans 9.1.

%%%%%%%%%%%%%%%%%%%%%%%%%%%%%%%%%%%%%%%%%%%%%%%%%%%%%

\subsection{Kleinste-Quadrate-Schätzer}
\subsection{Maximum-Likelihood-Schätzer}



% eine Seite = 40 Zeilen
- gegeben sei eine menge von unabhängigen stichproben, erhalten von verteilungsfunktionen fi\\
- in unserem regressionsfall zb sind die yi werte verteilt mit... sodass wir als verteilungsfunktionen ... erhalten\\
- die joint distribution aus den verteilungsfunktionen bezüglich aller yi ist eine funktion in abhängigkeit von theta und gegeben der stichprobe y und wird likelihood von theta genannt\\
- ziel der mle ist nun diese funktion zu maximieren\\
- das maximum ist durch theta mit ableitung 0 gegebne,da es aber sehr umständlich ist ein produkt abzuleiten, betrachtet man stattdessen den logarithmus der likelihood, welcher sich als summe der einzelnen logarithmen schreiben lässt\\
- da der logarithmus eine streng monoton steigende funktion ist lässt sich der mle nun auch als maximum der loglikelihood bestimmen\\
- ein sehr praktischer ansatz ist nun die bestimmung des maximums mittels numerischer verfahren, die nelder mead methode liefert ein robustes suchverfahren\\
- wichtige erkenntnisse über den mle sind nun dass unter sehr allgemeinen voraussetzungen mle multivariat normalverteilt ist mit mean theta0(wahrer wert) und varianz matrix V(theta0)\\
- wobei V sich als inverse der fischer informations matrix berechnen lässt, welches widerum der erwartungswert der hessischen matrix der loglik ist\\
- V kann durch V(mle) angenähert werden\\


\section{GoF-Tests}
\subsection{Kolmogorov-Smirnov Test}
\subsection{Anderson-Darling Test}
\subsection{Chi-Quadrat Test}


\section{Konfidenzintervalle}
\begin{equation}
\mathbb{P} \left( \theta_L \le \theta_0 \le \theta_U \right) \geq 1-\alpha
\end{equation}

\begin{equation}
\theta_L, \theta_U = \hat\theta \mp z_{\alpha/2}
\sqrt{
  V(\hat\theta)
}
\end{equation}

- wenn man ein parametrisches modell gefunden hat, welches die daten reptäsentieren soll, ist eine offensichtliche frage, wie akkurat diese schätzung ist\\
- oft wird ein interval bezgl des schätzers angegeben welches den wahren wert mit gewünschter wahrscheinlichkeit überdeckt\\
- solch ein interval heißt confidence intervall, klassische methoden zur bestimmung eines solchen intervals bauen auf asymtotischer theory und der sogenannten delta methode auf\\
- darstellung von ci zeigen\\
- ein konfindenz intervall für die varianz theta[1] in allen design punkten ist gegeben durch ... da ...\\
- eher interessiert uns allerdings ein confidence interval für die regressionsfunktion \\
- aufgrundlage von asymptotischer theorie (genauer die taylor expansion) und der deltamethode erhalten wir ein confidence interval für eta ...
- herleitung von russel zeigen...\\
- man beachte dass die ableitung und ... durch finite difference methoden berechnet werden könnne\\
- nachteil dieser herangehensweise sind ... -> bootstrap\\

\begin{equation}
\mathbb{P} \left(
  y_L(x) \le \eta(x, \theta_0) \le y_U(x)
\right) \geq 1-\alpha
\quad \forall x \in \mathbb{R}
\end{equation}

\begin{equation}
y_L(x), y_U(x) = \eta(x, \hat\theta) \mp 
z_{\alpha/2} 
\sqrt{
  \left( 
    \frac{\partial\eta(x, \theta)}{\partial\theta} 
  \right)_{\hat\theta}^T 
  V(\hat\theta) 
  \left( 
    \frac{\partial\eta(x, \theta)}{\partial\theta} 
  \right)_{\hat\theta}
}
\end{equation}


\section{Konfidenzbänder}
Banks, J. 1998. Handbook of simulation: 
- 7.4 Multivariate Estimation

- Quellen: Miller 1981\\
- interessanter für uns, als die schätzung von confidence intervallen für die einzelnen design punkte, ist eine schätzung von confidence intervallen, die für alle werte simultan gilt\\
- gesucht ist ein band welches mit gewünschter wahrscheinlichkeit die gesamte regressionsfunktion überdeckt\\
- beispiele von miller 1981 nenen ...\\
- eine einfaches und conservatives confidence band erhält man duch anwendung der taylor reihen expansion und asymptotischer theorie\\
- beispiel von russel zeigen ...\\
- für kleine n aller dings erhält man durch diesen ansatz oft fälschlich höhere werte für die konfidence als der eigentlich berechnete konfidence\\
- bootstrap kann in diesem fall helfen\\

\begin{equation}
\mathbb{P} \left(
  y_L(x) \le \eta(x, \theta_0) \le y_U(x) 
  \quad \forall x \in \mathbb{R}
\right) \geq 1-\alpha
\end{equation}

\begin{equation}
y_L(x), y_U(x) = \eta(x, \hat\theta) \mp 
\sqrt{
  \chi_p^2(a)
  \left( 
    \frac{\partial\eta(x, \theta)}{\partial\theta} 
  \right)_{\hat\theta}^T 
  V(\hat\theta) 
  \left( 
    \frac{\partial\eta(x, \theta)}{\partial\theta} 
  \right)_{\hat\theta}
}
\end{equation}


\section{Coverage Error}
- die qualität der confindence bereiche wir oft in form sogenannter coverage error beschrieben\\
- diese können durch asymptotische theorie bestimmt werden, auch für die bootstrap versionen\\
- coverage error ist in der regel O(1 / sqrt n) aber kann oft auf O(1 / n) durch balanced ci reduziert werden \\
- coverage error kommt hauptsächlich vom bias, da der effekt entgegengesetzte ist links und rechts von null hebt er sich auf, falls die ci balanciert werden\\
- coverage error kann als maß für den unterschied zwischen erreichter und gewünschter überdeckungswahrscheinlichkeit dienen\\



Sobald ein Regressionsmodell den Daten angepasst wurde ist es üblich, ein Konfidenz-Interval anzugeben, welches die Genauigkeit der Regressionsfunktion anzeigt. 

Für die einzelnen Zufallsvariablen ... sind solche Bereichsschätzer sehr einfach zu bestimmen. 

Der mittels Maximum-Likelihood-Methode geschätzte Parameter ... für die Standardabweichung liefert bereits das Konfidenz-Intervall ... 


\section{Resampling Verfahren}

\subsection{Jackknife}

\subsection{Bootstrap}
\begin{algorithm} 
  \caption{Basic-Sampling Methode} 
  \label{algo:sampling} 

  \begin{algorithmic}
  	\FOR{$j=0$ to $B$}
      \FOR{$i=0$ to $n$} 
		    \STATE ziehe ein Sample $y_{ij}$ von $F(.)$
		  \ENDFOR
			\STATE berechne die Statistik $s_j = s(y_j)$
    \ENDFOR  
  \end{algorithmic}
\end{algorithm} 

\begin{algorithm} 
  \caption{Bootstrap-Sampling Methode} 
  \label{algo:bootstrap} 

  \begin{algorithmic}
    \REQUIRE zufälliges Sample $y = (y_1, y_2, ...y_n)$ von $F(.)$
    \STATE erstelle die EDF $F_n(.|y)$
		\FOR{$j=0$ to $B$}
			\FOR{$i=0$ to $n$} 
			  \STATE ziehe ein Sample $y^*_{ij}$ von $F_n(.|y)$
			\ENDFOR
			\STATE berechne die Statistik $s^*_j = s(y^*_j)$
		\ENDFOR
		\STATE erstelle die EDF $G_n(.|s*)$ 
  \end{algorithmic}
\end{algorithm} 
























